\documentclass[10pt]{article}

\usepackage[utf8]{inputenc}
\usepackage{floatrow}

\usepackage{algorithm, algpseudocode}
\let\oldReturn\Return
\renewcommand{\Return}{\State\oldReturn}
\newcommand{\N}{\mathbb{N}}
\newcommand{\R}{\mathbb{R}}
\usepackage[T1]{fontenc}
\usepackage{enumitem}
\usepackage{hyperref}
\usepackage{scrextend}
\usepackage{amsmath}
\usepackage{amsfonts}
\usepackage{stmaryrd}
\usepackage{graphicx}
\usepackage{color}
\usepackage{listings}
\usepackage{wrapfig}
\usepackage[hmargin=1.25in,vmargin=1.25in]{geometry}

%title setup
\title{TD1 : l'ordonnanceur du noyau Linux}
\author{Romain PEREIRA}
\date{25/10/2018}

% table of contents setup
\renewcommand{\contentsname}{Sommaire}
\usepackage{etoolbox}
\patchcmd{\thebibliography}{\section*{\refname}}{}{}{}

\usepackage[utf8]{inputenc}
\usepackage[T1]{fontenc}
\usepackage[frenchb]{babel}

\setlength{\parindent}{0cm}
\setlength{\parskip}{1ex plus 0.5ex minus 0.2ex}
\newcommand{\hsp}{\hspace{20pt}}
\newcommand{\HRule}{\rule{\linewidth}{0.5mm}}

\hypersetup{
    colorlinks,
    citecolor=black,
    filecolor=black,
    linkcolor=blue,
    urlcolor=red
}

\lstset{language=C,
                basicstyle=\ttfamily,
                keywordstyle=\color{blue}\ttfamily,
                stringstyle=\color{red}\ttfamily,
                commentstyle=\color{cyan}\ttfamily,
                morecomment=[l][\color{magenta}]{\#}
}

\lstdefinelanguage{Markdown}{}

\begin{document}
    
    \begin{titlepage}
        \begin{sffamily}
            \begin{center}
                
                                
                \begin{figure}[h!]
                    \includegraphics[width=6cm]{ensiie.jpeg}
                \end{figure}

                \HRule \\[0.4cm]
                { \huge \bfseries Architecture d'un système d'exploitation\\[0.4cm] }
                \HRule \\[2.0cm]
                
                { \huge \bfseries TD2\\[0.5cm] }

                \textsc{\Large L'ordonnanceur du noyau Linux}\\[2.0cm]

                \vfill
                \begin{minipage}{0.4\textwidth}
                    \begin{flushleft} \large
                        \emph{Etudiant:} Romain \textsc{Pereira}\\
                    \end{flushleft}
                \end{minipage}
                \begin{minipage}{0.4\textwidth}
                    \begin{flushright} \large
                        \emph{Encadrants:}  M. \textsc{Loussert}\\
                                            M. \textsc{Taboada}
                    \end{flushright}
                \end{minipage}
                \\[2.0cm]
                {\large 25/10/2018}
            \end{center}
        \end{sffamily}
    \end{titlepage}
    
    \maketitle
    \tableofcontents
    \section{Préambule}
    
    Le but de ce TP est de comprendre le fonctionnement d'un ordonnanceur, en étudiant l'ordonnanceur du noyau Linux.

    \newpage
    \section{Fonctionnement de l'ordonnanceur}
        \paragraph{Q.1}
        Nos ordinateurs de bureau dispose généralement d'un seul processeur (avec 1, 2, 4 ou 8 coeurs d'execution).
        
        Lorsque plusieurs processus s'execute sur la machine, il faut donc qu'il
        y ait un méchanisme de partage du processeur : il est le seul point d'execution pour les processus.
    
        C'est la fonction principale de l'ordonnanceur.
        Il choisit dans quel ordre plusieurs processus peuvent être executé sur un même processeur.
        
        \textbf{N.B} : l'ordonnanceur est lui même un processus.
                
        L'execution de l'ordonnanceur peut avoir lieu pour répondre aux problèmes suivants:
        
        - Lorsqu'un processus se clone (\textit{fork()}) : lequel du père ou du fils doit avoir la priorité d'éxecution?
        
        - Lorsque le processus courant s'arrête : quel processus doit prendre la main ensuite
        
        - Lorsqu'un processus bloque sur une entrée/sortie, ou qu'il est en attente : doit t'il continuer de bloquer le processeur alors qu'il ne travaille pas?
        
        - Lorsqu'une entrée/sortie est interrompu : peut être que le processus a fini son job, ou bien c'est un signal indiquant qu'un processus précèdemment en attente entrée/sortie peut maintenant effectué son travail.
        
        
        Aussi, on distingue 2 grandes politiques décidant de l'execution de l'ordonnanceur.

        - non-pre-emptive : l'ordonnanceur ne tourne que si le processus utilisateur s'arrête, bloque, ou demande explicitement à changer de processus.
        Si le processus ne rends pas la main pendant 100 ans, l'ordonanneur attendra impassiblement pendant 100 ans.

        \label{preempt}
        - pre-emptive : Chaque processus à un temps d'execution fixe.
        L'ordonnanceur est executé à intervalle régulier, et détermine le prochain prochain processus qui doit être executé sur le prochaine 'créneau'.


        \paragraph{Q.2}
        La fonction \textit{sched\_yield} passe la main sur le processeur d'un processus à un autre
        (ou d'un processus vers lui même s'il reste le plus 'prioritaire').
        
        S'il n'y a pas d'autre processus devant être executer, cette fonction s'arrête.
        
        \paragraph{Q.3}
        Code modifié
        \lstset{language=C}
\begin{lstlisting}[frame=single]
[...]
printk("current=%p\n", current);
schedule();
printk("current=%p\n", current);
[...]
\end{lstlisting}

        Compilation de l'image linux modifié et déploiement
\lstset{language=bash}
\begin{lstlisting}[frame=single]
cd ~/debian_kernel/linux-4.9.30/
make bindeb-pkg
dpkg -i ../linux-image-4.9.30_4.9.30-2_amd64.deb
reboot
\end{lstlisting}

Programme utilisateur 'main.c'
\lstset{language=C}
\begin{lstlisting}[frame=single]
int main() {
    sched_yield();
    return 0;
}
\end{lstlisting}

Compile et lance le programme utilisateur
\begin{lstlisting}[frame=single]
gcc main.c
./a.out
dmesg
\end{lstlisting}

Sortie dmesg
\begin{lstlisting}[frame=single]
[  170.029953] current=ffff9337bbe47140
[  170.029956] current=ffff9337bbe47140
\end{lstlisting}

=> Le pointeur 'current' reste inchangé
Ceci est dû au fait que peu de processus tourne sur la machine virtuelle.
Le processus prioritaire reste le même entre 2 appels de 'schedule()'

En lançant d'autres processus sur la machine, la valeur du pointeur aurait changé


\paragraph{Q.4}
On génère d'abord les 'ctags' pour naviguer plus facilement dans le code source.

\lstset{language=bash}
\begin{lstlisting}[frame=single]
cd ~/debian_kernel/linux-4.9.30/
ctags -R .
\end{lstlisting}


Lecture de 'core.c'
\lstset{language=C}
\begin{lstlisting}[frame=single]
ligne  profondeur/schedule()  execution
4898  0  schedule();

3454  1  sched_submit_work(task);
3438  2  ...

//debut boucle : desactive le préemption
3456  1  preempt_disable();

3457  1  __schedule(false);
3342  2  cpu_rq(); // recuperes la 'run queue' du CPU
3343  2  prev = rq->curr; // enregistre le processus courant dans la run queue
3391  2  next = pick_next_task(rq, prev, cookie); // recuperes la task suivante

// réactive la préemption
3457  1  sched_preempt_enable_no_resched();

3459  1  need_resched();
// tant que 'need_resched()' renvoit 'vrai',
// on boucle sur partir de preempt_disable()
\end{lstlisting}
        
\paragraph{Q.5} La fonction principal de l'ordonnanceur est 'schedule()'

\lstset{language=C}
\begin{lstlisting}[frame=single]
// choisit le prochain processus à être executé
// avant l'appel de cette fonction, un processus tourne
schedule();
// apres l'appel, un autre processus peut tourner
\end{lstlisting}
        
\paragraph{Q.6}
La préemption est un mecanisme d'ordonnancement qui consiste à donner un temps maximal d'execution à un processus,
au dela duquelle l'algorithme d'ordonnancement sera relancé (c.f \ref{preempt} )

Si la préemption est active lors de l'execution de l'ordonnanceur, ...
 
    \newpage
    \section{Références}
    \begin{thebibliography}{}

        \bibitem{modernos}\label{modernos}
        Modern Operating Systems - Andrew S. Tanenbaum, Herbert Bos (page 149-166)\newline
        \href{https://github.com/concerttttt/books/blob/master/Modern Operating Systems 4th Edition--Andrew Tanenbaum.pdf}
        {\textit{https://github.com/concerttttt/books/}}\newline
        
    \end{thebibliography}
    
\end{document}
