\documentclass[10pt]{article}

\usepackage[utf8]{inputenc}
\usepackage{floatrow}

\usepackage{algorithm, algpseudocode}
\let\oldReturn\Return
\renewcommand{\Return}{\State\oldReturn}
\newcommand{\N}{\mathbb{N}}
\newcommand{\R}{\mathbb{R}}
\usepackage[T1]{fontenc}
\usepackage{enumitem}
\usepackage{hyperref}
\usepackage{scrextend}
\usepackage{amsmath}
\usepackage{amsfonts}
\usepackage{stmaryrd}
\usepackage{graphicx}
\usepackage{color}
\usepackage{listings}
\usepackage{wrapfig}
\usepackage[hmargin=1.25in,vmargin=1.25in]{geometry}

% table of contents setup
\renewcommand{\contentsname}{Sommaire}
\usepackage{etoolbox}
\patchcmd{\thebibliography}{\section*{\refname}}{}{}{}

\usepackage[utf8]{inputenc}
\usepackage[T1]{fontenc}
\usepackage[frenchb]{babel}

\setlength{\parindent}{0cm}
\setlength{\parskip}{1ex plus 0.5ex minus 0.2ex}
\newcommand{\hsp}{\hspace{20pt}}
\newcommand{\HRule}{\rule{\linewidth}{0.5mm}}

\hypersetup{
    colorlinks,
    citecolor=black,
    filecolor=black,
    linkcolor=blue,
    urlcolor=red
}

\lstset{language=C,
                basicstyle=\ttfamily,
                keywordstyle=\color{blue}\ttfamily,
                stringstyle=\color{red}\ttfamily,
                commentstyle=\color{cyan}\ttfamily,
                morecomment=[l][\color{magenta}]{\#}
}

\begin{document}
    
    \begin{titlepage}
        \begin{sffamily}
            \begin{center}

                \begin{figure}[h!]
                    \includegraphics[width=6cm]{ensiie.jpeg}
                \end{figure}

                \HRule \\[0.8cm]
                { \huge \bfseries Architecture d'un système d'exploitation (UE S3) } \\[0.4cm]
                \HRule \\[2.0cm]
                
                { \huge \bfseries TP4 } \\[0.5cm]

                \textsc{\Large Introduction à la sécurité informatique en environnement HPC}\\[2.0cm]

                \vfill
                \begin{minipage}{0.4\textwidth}
                    \begin{flushleft} \large
                        \emph{Etudiant:} Romain \textsc{Pereira}\\
                    \end{flushleft}
                \end{minipage}
                \begin{minipage}{0.4\textwidth}
                    \begin{flushright} \large
                        \emph{Encadrant:}  M. F. \textsc{Combeau}
                    \end{flushright}
                \end{minipage}
                \\[2.0cm]
                {\large 17/11/2018}
            \end{center}
        \end{sffamily}
    \end{titlepage}
    
    \tableofcontents

    \section{Réponses aux questions 1 à 22 (faites en cours)}

        \paragraph{4.} Pour se connecter, on a utilisé l'authentification système, à savoir, les \textbf{Pluggable Authentication Modules (PAM)}
        
        \paragraph{5.} Le mot de passe \textit{root} est stocké dans le fichier '/etc/shadow' sous forme d'un hachage. Le mot de passe est haché avant d'être stocké car il est stocké localement (chaque machine connecté au réseau possède le fichier '/etc/shadow' localement).
        
        \paragraph{6.} Je ne peux pas déterminer facilement le mot de passe de \textit{Bob}, car il est également stocké sous forme de hash. Cependant, à l'aide d'une méthode de brute force, on pourrait eventuellement déterminer des chaîne de caractères qui ont le même hash que celui de \textit{Bob}. Le mot de passe de \textit{Bob} serait donc potentiellement l'une de ces chaînes de caractères, ou pas.
        
        On ne peut pas déterminer le mot de passe d'\textit{Alice}, car elle n'en a pas.
        En effet, le champ devant correspondre à son mot de passe dans le fichier '/etc/shadow' est \textbf{*} (ou \textbf{!!}).
        Les fonctions de hachages utilisées par les PAM ne donneront jamais des hashs avec ces caractères ('*' et '!').
        En spécifiant des caractères impossibles à obtenir par hachage, on désactive de façon indirecte l'authentification par mot de passe pour l'utilisateur : jamais un mot de passe ne donnera ce hash.
        L'utilisateur est en quelque sorte 'banni'.

        \paragraph{7.} Si l'on passe cette ligne de \textbf{sufficient} à \textbf{required}, alors \textit{root} ne pourra plus se logger s'authentifier.
        En effet, l'identifiant utilisateur (\textit{uid}) de \textit{root} vaut 0.
        A la ligne suivante, on refuse l'authentification des utilisateurs dont l'uid est inférieur à 1000, le test ne passera donc pas.
        est donc inférieur à 1000 : les tests suivant ne passeront pas.
        
        Finalement, \textit{root} ne pourra plus se logger (à cause de la ligne \textit{pam\_deny.so})

        
        \paragraph{8.} Oui, on arrive à se connecter sur le compte de \textit{Bob} à partir du compte \textit{root}
        
        \paragraph{9.}

\begin{lstlisting}[language=bash,caption={bash version}]
> less /var/log/secure
[...]
Accepted password for bob form ::1 port 47404 ssh2
pam_unix(sshd:session) : session opened for user bob by (uid=0)
[...]
\end{lstlisting}

La connexion sur le compte \textit{bob} à partir de l'utilisateur \textit{root} (uid=0) est bien tracée dans les logs.

        \paragraph{10.}
        
        \paragraph{11.}
        
        \paragraph{12.}
        
        \paragraph{13.}
        
        \paragraph{14.}
        
        \paragraph{15.}
        
        \paragraph{16.}
        
        \paragraph{17.}
        
        \paragraph{18.}
        
        \paragraph{19.}
        
        \paragraph{20.}
        
        \paragraph{21.}
        
        \paragraph{22.}

    \section{Réponses aux questions XX à YY (bonus)}

\end{document}
