\documentclass[10pt]{article}

\usepackage[utf8]{inputenc}
\usepackage{floatrow}

\usepackage{algorithm, algpseudocode}
\let\oldReturn\Return
\renewcommand{\Return}{\State\oldReturn}
\newcommand{\N}{\mathbb{N}}
\newcommand{\R}{\mathbb{R}}
\usepackage[T1]{fontenc}
\usepackage{enumitem}
\usepackage{hyperref}
\usepackage{scrextend}
\usepackage{amsmath}
\usepackage{amsfonts}
\usepackage{stmaryrd}
\usepackage{graphicx}
\usepackage{color}
\usepackage{listings}
\usepackage{wrapfig}
\usepackage[hmargin=1.25in,vmargin=1.25in]{geometry}

% table of contents setup
\renewcommand{\contentsname}{Sommaire}
\usepackage{etoolbox}
\patchcmd{\thebibliography}{\section*{\refname}}{}{}{}

\usepackage[utf8]{inputenc}
\usepackage[T1]{fontenc}
\usepackage[frenchb]{babel}

\setlength{\parindent}{0cm}
\setlength{\parskip}{1ex plus 0.5ex minus 0.2ex}
\newcommand{\hsp}{\hspace{20pt}}
\newcommand{\HRule}{\rule{\linewidth}{0.5mm}}

\hypersetup{
    colorlinks,
    citecolor=black,
    filecolor=black,
    linkcolor=blue,
    urlcolor=red
}

\lstset{language=C,
                basicstyle=\ttfamily,
                keywordstyle=\color{blue}\ttfamily,
                stringstyle=\color{red}\ttfamily,
                commentstyle=\color{cyan}\ttfamily,
                morecomment=[l][\color{magenta}]{\#}
}

\begin{document}
    
    \begin{titlepage}
        \begin{sffamily}
            \begin{center}

                \begin{figure}[h!]
                    \includegraphics[width=6cm]{ensiie.jpeg}
                \end{figure}

                \HRule \\[0.8cm]
                { \huge \bfseries Architecture d'un système d'exploitation (UE S3) } \\[0.4cm]
                \HRule \\[2.0cm]
                
                { \huge \bfseries TP4 } \\[0.5cm]

                \textsc{\Large Introduction à la sécurité informatique en environnement HPC}\\[2.0cm]

                \vfill
                \begin{minipage}{0.4\textwidth}
                    \begin{flushleft} \large
                        \emph{Etudiant:} Romain \textsc{Pereira}\\
                    \end{flushleft}
                \end{minipage}
                \begin{minipage}{0.4\textwidth}
                    \begin{flushright} \large
                        \emph{Encadrant:}  M. F. \textsc{Combeau}
                    \end{flushright}
                \end{minipage}
                \\[2.0cm]
                {\large 17/11/2018}
            \end{center}
        \end{sffamily}
    \end{titlepage}
    
    \tableofcontents

    \section{Réponses aux questions 1 à 22 (faites en cours)}

        \paragraph{4.} Pour se connecter, on a utilisé l'authentification système, à savoir, les \textbf{Pluggable Authentication Modules (PAM)}
        
        \paragraph{5.} Le mot de passe \textit{root} est stocké dans le fichier '/etc/shadow' sous forme d'un hachage. Le mot de passe est haché avant d'être stocké car il est stocké localement (chaque machine connecté au réseau possède le fichier '/etc/shadow' localement).
        
        \paragraph{6.} Je ne peux pas déterminer facilement le mot de passe de \textit{Bob}, car il est également stocké sous forme de hash. Cependant, à l'aide d'une méthode de brute force, on pourrait eventuellement déterminer des chaîne de caractères qui ont le même hash que celui de \textit{Bob}. Le mot de passe de \textit{Bob} serait donc potentiellement l'une de ces chaînes de caractères, ou pas.
        
        On ne peut pas déterminer le mot de passe d'\textit{Alice}, car elle n'en a pas.
        En effet, le champ devant correspondre à son mot de passe dans le fichier '/etc/shadow' est \textbf{*} (ou \textbf{!!}).
        Les fonctions de hachages utilisées par les PAM ne donneront jamais des hashs avec ces caractères ('*' et '!').
        En spécifiant des caractères impossibles à obtenir par hachage, on désactive de façon indirecte l'authentification par mot de passe pour l'utilisateur : jamais un mot de passe ne donnera ce hash.
        L'utilisateur est en quelque sorte 'banni'.

        \paragraph{7.} Si l'on passe cette ligne de \textbf{sufficient} à \textbf{required}, alors \textit{root} ne pourra plus se logger s'authentifier.
        En effet, l'identifiant utilisateur (\textit{uid}) de \textit{root} vaut 0.
        A la ligne suivante, on refuse l'authentification des utilisateurs dont l'uid est inférieur à 1000, le test ne passera donc pas.
        est donc inférieur à 1000 : les tests suivant ne passeront pas.
        
        Finalement, \textit{root} ne pourra plus se logger (à cause de la ligne \textit{pam\_deny.so})

        
        \paragraph{8.} Oui, on arrive à se connecter sur le compte de \textit{Bob} à partir du compte \textit{root}
        
        \paragraph{9.}
        La connexion sur le compte \textit{bob} à partir de l'utilisateur \textit{root} (uid=0) est bien tracée dans les logs.

\begin{lstlisting}[language=bash]
> less /var/log/secure
[...]
Accepted password for bob form ::1 port 47404 ssh2
pam_unix(sshd:session) : session opened for user bob by (uid=0)
[...]
\end{lstlisting}

        \paragraph{10.}
        \textbf{requisite} et \textbf{required} rende obligatoire le succès de l'entrée PAM.
        
        Avec \textbf{requisite}, le processus d'authentification continue quand même les instructions suivantes, même si l'authentification a échoué.
        
        Avec \textbf{required}, le processus d'authentification s'arrête si l'instruction renvoie faux.
        
        Danns cet exemple, rien ne change en passant de \textbf{requisite} à \textbf{required} car dans tous les cas, l'authentification s'arrête au 'deny' qui suit.

        En passant de \textbf{requisite} à \textbf{sufficient}, si la ligne 'pam\_succeed\_if.so' renvoie vrai, alors l'authentification est vérifié, renvoie vrai.
        
        Une entrée \textbf{sufficient} est 'suffisante' : sa validité seule permet de valider l'authentification.
        
        La ligne avec le 'deny' ne sera donc pas appelé si le test est passé : l'utilisateur pourra s'authentifier pourvu qu'il ait un \textbf{'uid' supérieur ou égal à 1000}.
 
        \paragraph{11.} Alice n'a pas de mot de passe (cf '!!' et '*')
        
\begin{lstlisting}[language=bash]
> ssh alice@localhost
Enter passphrase for key '/root/.ssh/id_rsa'
\end{lstlisting}

Donc ssh utilise son propre système d'authentification, et non pas les PAM systèmes.
La clef privé ssh (RSA) a été chiffré à l'aide d'un mot de passe (pour éviter le vol de clef)

Une fois connecté sous Alice, on a la clef dans '.ssh/authorized\_keys'

Logs:
\begin{lstlisting}[language=bash]
sshd : Accepted publickey for alice from ::1 port 47410 ssh2: RSA SHA256:/JdhKRDYT.....
sshd : pam_unix(sshd:session): session opened for user alice by (uid=0)
\end{lstlisting}

        \paragraph{12.} Protection '775' ('002' est le complément à 8 du masque octal)

\begin{lstlisting}[language=bash]
> drwxrwxr-x 2 bob 4096 19 nov. 15:22
> umask
'0002'
\end{lstlisting}

        \paragraph{13.}
        
Oui on a le droit de créer, car on est sous root (root à le droit d'écrire partout)
\begin{lstlisting}[language=bash]
> -rw-r--r--. 1 root    0 19 nov. 15:52 toto
\end{lstlisting}

Ce ficher appartient à root
Linux applique le masque octal (umask), mais par défaut, il est non executable par mesure de sécurité.

\paragraph{14.} Le système notifie qu'on a pas les droits d'écriture sur le fichier (entant que 'bob')
Mais le fichier a été supprimé, car 'bob' à le droit d'écriture dans le dossier

\begin{lstlisting}[language=bash]
> rm toto
'rm : supprimer fichier vide (protégé en écriture) " toto " ? y
\end{lstlisting}
        
        \paragraph{15.} Il suffit d'utiliser l'utilitaire 'chmod'
        
\begin{lstlisting}[language=bash]
> chmod 1775 test
> drwsr-sr-t. 1 bob    0 19 nov. 15:52 test
\end{lstlisting}

        \paragraph{16.} Oui, on a toujours le droit, et le fichier appartient à 'root'
\begin{lstlisting}[language=bash]
> -rw-r--r--. 1 root    0 19 nov. 15:52 toto
\end{lstlisting}
 
        \paragraph{17.}
        Oui on a toujours le droit.
        
        \paragraph{18.} Il suffit d'utilliser l'utilitaire 'chown'
 \begin{lstlisting}[language=bash]
 > chown root test
 > drwsr-sr-t. 1 root    0 19 nov. 15:52 test
\end{lstlisting}

        \paragraph{19.}
        Non \textbf{bob} ne peut pas l'effacer, car il y a le sticky bit (le 't') sur le repertoire.
        Il faut être propriétaire du dossier pour supprimer un fichier.
        Dans ce cas, \textbf{root} est le propriétaire du dossier, donc \textbf{bob} ne peut pas effacer de fichiers.
        
        \paragraph{20.}
        \textit{/bin/passwd} : modifie des fichiers auxquelles seul root à accès (ex: \textit{/etc/shadow})
        
        \paragraph{21.}
        \textbf{Bob} peut partager son fichier en utilisant les ACL.
        
        Pour pouvoir parcourir le dossier
\begin{lstlisting}[language=bash]
> setfacl -m u:alice:rx dossier
\end{lstlisting}

Pour pouvoir ecrire dans le fichier du dossier
\begin{lstlisting}[language=bash]
> setfacl -m u:alice:rw pour_alice
\end{lstlisting}
 
 \paragraph{22.}
 On remarque que l'utilisateur 'sftp' est ch-rooté dans '/home/sftp'

Lors d'une connection ssh:

\begin{lstlisting}[language=bash]
This service allows sftp connections only.
Connection to localhost closed.
\end{lstlisting}

(le service n'autorise que des connections sftp)

Lors d'une connection sftp:

\begin{lstlisting}[language=bash]
'Connected to localhost'
\end{lstlisting}

Avec l'utilisateur \textbf{bob}, le chroot est '/', alors qu'avec \textbf{sftp}, le chroot avec '/home/stfp'


    \section{Réponses aux questions 23 à 32 (bonus)}
    
     \paragraph{23.} Oui, l'utilisateur root put prendre l'identité de n'importe quel utilisateur, \textbf{sans même} avoir à entrer un mot de passe (pas d'authentification requise).

\begin{lstlisting}[language=bash]
> su bob
\end{lstlisting} 
    
    \textbf{``su bob''} change l'utilisateur courant, en gardant le même shell (et la plupart des variables d'environnement)
    
    \textbf{``su - bob''} change l'utilisateur courant et crée un nouveau shell, en redefinissant la plupart des variables d'environnement.
    
     \paragraph{24.} Oui, \textbf{bob} peut prendre l'identité de \textbf{root}, mais seulement s'il connait
     le mot de passe de \textbf{root} : une authentification est demandée.
     
\begin{lstlisting}[language=bash]
[bob@localhost ~] > su root
Mot de passe:
[root@localhost bob] > _
\end{lstlisting}
    
    \paragraph{25.} La commande 'su' peut être desactivé pour tous les utilisateurs qui ne font pas parti du groupe 'wheel'. Il suffit de configurer les PAM, en éditant la configuration dans \textbf{/etc/pam.d/su}, et en décommentant la ligne 15 suivante :
    
\begin{lstlisting}[language=bash]
  8 # Uncomment this to force users to be a member of group root
  9 # before they can use `su'. You can also add "group=foo"
 10 # to the end of this line if you want to use a group other
 11 # than the default "root" (but this may have side effect of
 12 # denying "root" user, unless she's a member of "foo" or explicitly
 13 # permitted earlier by e.g. "sufficient pam_rootok.so").
 14 # (Replaces the `SU_WHEEL_ONLY' option from login.defs)
 15 auth       required   pam_wheel.so
\end{lstlisting}

    \paragraph{26.} Oui, \textbf{bob} peut passer \textbf{root} via ssh s'il connait le mot de passe.
    
    Pour l'en empêcher, on peut configurer \textbf{sshd} en éditant le fichier \textbf{/etc/ssh/sshd\_config}
    
    Il suffit de modifier la ligne 38:
    
\begin{lstlisting}[language=bash]
[...]
 37 #LoginGraceTime 2m
 38 #PermitRootLogin yes
 39 #StrictModes yes
[...]
\end{lstlisting}

    par:
    
\begin{lstlisting}[language=bash]
[...]
 37 #LoginGraceTime 2m
 38 PermitRootLogin no
 39 #StrictModes yes
[...]
\end{lstlisting}

Puis il faut redémarrer le serveur ssh pour que la configuration soit bien prise en compte
\begin{lstlisting}[language=bash]
[root@localhost ~]# /etc/init.d/sshd restart
Stopping sshd: [ OK ]
Starting sshd: [ OK ]
\end{lstlisting}


    \paragraph{27.} Non, \textbf{bob} n'est pas autorisé à utiliser la commande \textbf{sudo}
    
\begin{lstlisting}[language=bash]
[bob@localhost ~]# sudo ls
Mot de passe:
bob n'apparaît pas dans la liste des sudoers. Cet évènement sera signalé.
\end{lstlisting}

Le fichier de configuration de la liste des 'sudoers' se situe dans '/etc/sudoers'

On peut ajouter \textbf{bob} à la liste en ajoutant cette ligne au fichier de configuration:

\begin{lstlisting}[language=bash]
bob ALL=(ALL) ALL
\end{lstlisting}

Par défaut, la configuration '/etc/sudoers' autorise tous les membres du groupe 'sudo' à utiliser la commande \textbf{sudo}.

On peut donc autoriser \textbf{bob} à utiliser 'sudo' en l'ajoutant au groupe 'sudo':

\begin{lstlisting}[language=bash]
[root@localhost ~]# sudo adduser bob sudo
\end{lstlisting}

    \paragraph{28.}
    Oui, on peut autoriser \textbf{bob} à utiliser certaines commandes uniquement en tant que root.
    
    Il suffit d'éditer le fichier '/etc/sudoers' et d'y ajouter/modifier la ligne concernant \textbf{bob}.
    
    On peut aussi l'autoriser à éxecuter des commandes en tant que \textbf{root} sans qu'il n'ait à rentrer son mot de passe. Exemple:
    
\begin{lstlisting}[language=bash]
bob ALL=(ALL) NOPASSWD: cat /var/log/secure
\end{lstlisting}
    
    \paragraph{29.}
    Pour récuperer le shell d'Alice, une façon de faire est de récupérer les informations d'Alice via \textbf{getent}.
    La 7ème entrée correspond au shell.
    
\begin{lstlisting}[language=bash]
[root@localhost ~]# getent passwd | grep alice | cut -d : -f 7
/bin/bash
\end{lstlisting}

    \paragraph{30.}
    Si l'on essaye de modifier le shell d'\textbf{alice} par '/bin/false', en tant qu'utilisateur (\textbf{alice} par exemple),
    cela nous est refusé.
\begin{lstlisting}[language=bash]
[alice@localhost ~]# chsh -s /bin/false alice
chsh: "/bin/false" n'apparait pas dans /etc/shells
Utiliser chsh -l pour afficher la liste.
\end{lstlisting}
    
    Par contre, si on essaye de changer le shell d'\textbf{alice} entant que \textbf{root}:
    
\begin{lstlisting}[language=bash]
[root@localhost ~]# chsh -s /bin/false alice
chsh: Avertissement, "/bin/false" n'apparait pas dans /etc/shells
L'interpréteur a ete modifie.
\end{lstlisting}

Autrement dit, le shell a bien été changé, mais nous avons eu un message d'avertissement car le binaire '/bin/false' n'est pas enregistré en tant que shell valide sur le système.

L'executable '/sbin/nologin' est un shell enregistré, on peut donc le définir en tant que shell sans avertissement.

Le comportement final est le même: l'utilisateur n'a pas de shell.


    \paragraph{31.}
    Si l'on change le shell d'\textbf{alice} par '/bin/rbash', le shell par défaut d'\textbf{alice} sera un bash restreint.
    Ce bash à l'utilisation restreinte ajoute une couche de sécurité.
    
    
    \paragraph{32.}
    '/bin/rbash' est un lien symbolique vers le binaire 'bash'.
        
\begin{lstlisting}[language=bash]
[root@localhost ~]# ls -la /bin/rbash
lrwxrwxrwx. 1 root root 9 5 nov. 2017 /bin/rbash -> /bin/bash
\end{lstlisting}
        
    Autrement dit, le comportement du programme 'bash' varie en fonction de son nom (argv[0]).
    
\end{document}
