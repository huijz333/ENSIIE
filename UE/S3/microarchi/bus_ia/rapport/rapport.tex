\documentclass[10pt]{article}

\usepackage[utf8]{inputenc}
\usepackage{floatrow}

\usepackage{algorithm, algpseudocode}
\let\oldReturn\Return
\renewcommand{\Return}{\State\oldReturn}
\newcommand{\N}{\mathbb{N}}
\newcommand{\R}{\mathbb{R}}
\usepackage[T1]{fontenc}
\usepackage{enumitem}
\usepackage{hyperref}
\usepackage{scrextend}
\usepackage{amsmath}
\usepackage{amsfonts}
\usepackage{stmaryrd}
\usepackage{graphicx}
\usepackage{color}
\usepackage{listings}
\usepackage{wrapfig}
\usepackage[hmargin=1.25in,vmargin=1.25in]{geometry}

% table of contents setup
\renewcommand{\contentsname}{Sommaire}
\usepackage{etoolbox}
\patchcmd{\thebibliography}{\section*{\refname}}{}{}{}

\usepackage[utf8]{inputenc}
\usepackage[T1]{fontenc}
\usepackage[frenchb]{babel}

\setlength{\parindent}{0cm}
\setlength{\parskip}{1ex plus 0.5ex minus 0.2ex}
\newcommand{\hsp}{\hspace{20pt}}
\newcommand{\HRule}{\rule{\linewidth}{0.5mm}}

\hypersetup{
    colorlinks,
    citecolor=black,
    filecolor=black,
    linkcolor=blue,
    urlcolor=red
}

\lstset{language=C,
                basicstyle=\ttfamily,
                keywordstyle=\color{blue}\ttfamily,
                stringstyle=\color{red}\ttfamily,
                commentstyle=\color{cyan}\ttfamily,
                morecomment=[l][\color{magenta}]{\#}
}

\begin{document}
    
    \begin{titlepage}
        \begin{sffamily}
            \begin{center}

                \begin{figure}[h!]
                    \includegraphics[width=6cm]{ensiie.jpeg}
                \end{figure}

                \HRule \\[0.8cm]
                { \huge \bfseries Micro-architecture (UE S3) } \\[0.4cm]
                \HRule \\[2.0cm]
                
                { \huge \bfseries Dossier de Projet } \\[0.5cm]

                \textsc{\Large Serpentin 7-segment programmable (FPGA)}\\[2.0cm]

                \vfill
                \begin{minipage}{0.4\textwidth}
                    \begin{flushleft} \large
                        \emph{Etudiant:} Afizullah \textsc{Rahmany}\\
                        \emph{Etudiant:} Romain \textsc{Pereira}\\
                    \end{flushleft}
                \end{minipage}
                \begin{minipage}{0.4\textwidth}
                    \begin{flushright} \large
                        \emph{Enseignant:}  M. \textsc{Augé}
                    \end{flushright}
                \end{minipage}
                \\[2.0cm]
                {\large 25/10/2018}
            \end{center}
        \end{sffamily}
    \end{titlepage}
    
    \tableofcontents
    \section{Préambule}
    Ce projet a été réalisé dans le cadre de nos études à l'ENSIIE (Ecole National Supérieur d'informatique pour l'industrie et l'entreprise) d'Evry.
    
    Ce projet est l'aboustissement de nos cours en Micro-architecture.
    
    Nous programmions sur un FPGA d'Altera (gamme Cyclone), en VHDL et à l'aide du logiciel Quartus.
    
    L'objectif a été de réalisé un serpentin programmable, et affichable sur un 7-segment.

    \newpage
    \section{Introduction}

    \newpage
    \section{Manuel utilisateur}
    
    \newpage
    \section{Présentation générale}
   
    \newpage
    \section{Description matérielle}
    
    \newpage
    \section{Présentation générale}
        \subsection{Schéma général}
        \subsection{Description des blocs}
        
    \newpage
    \section{Implémentation du bloc ????}
    
    
\end{document}
