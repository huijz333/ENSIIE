\documentclass[10pt]{article}

\usepackage[utf8]{inputenc}
\usepackage{floatrow}

\usepackage{algorithm}
\usepackage{algorithmic}
\usepackage[T1]{fontenc}
\usepackage{enumitem}
\usepackage{hyperref}
\usepackage{graphicx}
\usepackage{color}
\usepackage{listings}
\usepackage{wrapfig}
\usepackage{amsfonts}
\usepackage{amsmath}
\usepackage{mathtools}
\usepackage[hmargin=1.25in,vmargin=1.25in]{geometry}

%title setup
\title{MPM2 Projet Mathématique 2017-2018}
\author{
			Etienne TAILLEFER DE LAPORTALIERE \\
			Romain PEREIRA\\
			Cyril PIQUET
}
\date{13/02/2018}

% table of contents setup
\renewcommand{\contentsname}{Sommaire}
\usepackage{etoolbox}
\patchcmd{\thebibliography}{\section*{\refname}}{}{}{}

\hypersetup{
    colorlinks,
    citecolor=black,
    filecolor=black,
    linkcolor=black,
    urlcolor=red
}

\begin{document}
	\maketitle
	\tableofcontents

	\section*{Préambule}

		Ce projet est réalisé dans le cadre de nos études à l'ENSIIE.
		L'objectif est de modéliser un marché financier et de déterminer les prix et la couverture d'option européenne.

	\newpage
	\section{Modèle de Cox-Ross-Rubinstein}
		On a \( \boxed{l = 2^N} \) trajectoire possible pour l'évolution du prix de l'actif risqué, entre l'instant initial 1 et l'instant final N.\newline
		On a \( \boxed{N + 1}   \) valeurs possibles pour \( S^{N}_{t_N} \)\newline
		\newline
		INSERT SCHEMA ICI ILLUSTRANT LES 2 VALEURS PRECEDENTES
		\newline
		
		\textbf{1.} \(\mathbb{Q}\) est la probabilité risque-neutre, vérifiant: \( \mathbb{E}_{\mathbb{Q}}[T^{(N)}_{1}] = 1 + r_N \)
		\newline
		On note: \( q_n = \mathbb{Q}(T^{(N)}_1 = 1 + h_N) \)
		Donc:
		\begin{align}
			\begin{split}
				\mathbb{E}_{\mathbb{Q}}[T^{(N)}_1] &= \sum_{t \in T^{(N)}_{1}(\Omega)}x 	\mathbb{P}(T^{(N)}_{1} = t) \\
										\qquad &= (1 + h_n) q_N + (1 + b_n) (1 - q_n) \\
										\qquad &= 1 + r_N \\					
										\qquad \Rightarrow \Aboxed{q_N &= \frac{r_N - b_N}{h_N - b_N}}
			\end{split}
		\end{align}
		
		\textbf{2.} Soit \(   p_(N) = \frac{1}{(1 + r_N)^N} \mathbb{E}_{\mathbb{Q}}[f(S^{(N)}_{t_N})]   \),
		le prix de l'option qui paye \( f(S^{(N)}_{t_N} \). On a:
		\begin{align}
			\begin{split}
				\bullet \qquad p_{(0)} 	&= f(s) \\
				\bullet \qquad p_{(1)} 	&= \frac{1}{(1 + r_N)^1} \mathbb{E}_{\mathbb{Q}}[f(S^{(N)}_{t_1})] \\
										&= \frac{1}{(1 + r_N)^1} \mathbb{E}_{\mathbb{Q}}[f(T^{(N)}_{1} S^{(N)}_{t_0})] \\
										&= \frac{1}{1 + r_N} [ q_N f[(1 + h_N)  s] + (1 - q_N) f[(1 + b_N)  s)] ] \\
										&= [...] \text{ (en se basant sur l'arbre de la figure 1)}\\
				\bullet \qquad p_{(N)} 	&= \frac{1}{(1 + r_N)^N} \sum^{N}_{k = 0} {N\choose k} q_N^k (1 - q_N)^{N - k} f((1 + h_N)^k (1 + b_N)^{N - k})
			\end{split}
		\end{align}
	
		\subsection{Premier pricer}
		\subsection{Deuxième pricer}
		\subsection{Comparaison}
		\subsection{La couverture}
		
	\section{Modèle de Black-Scholes}
		\subsection{Le modèle}
		\subsection{Le pricer par la méthode de Monte-Carlo}
		\subsection{Le pricer par formule fermée}
		
	\section{Convergence des prix}
	
	\section{EDP de Black-Scholes}
		
	\newpage
	\section{Références}
		\begin{thebibliography}{}
		\end{thebibliography}

\end{document}
