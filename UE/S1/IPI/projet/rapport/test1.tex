\documentclass[a4paper,10pt]{article}
\usepackage[utf8]{inputenc}
\usepackage[T1]{fontenc}
\usepackage[french]{babel}
\frenchbsetup{StandardLists=true} % à inclure si on utilise \usepackage[french]{babel}
\usepackage{enumitem}
\usepackage{amssymb}

%title setup
\title{Projet IPI: chemins de poids minimum}
\author{Romain PEREIRA}
\date{04/12/2017}

% table of contents setup
\renewcommand{\contentsname}{Sommaire}

\usepackage{hyperref}
\hypersetup{
    colorlinks,
    citecolor=black,
    filecolor=black,
    linkcolor=blue,
    urlcolor=black
}

\begin{document}
  \maketitle
  \tableofcontents

  \section*{Préambule}
    Ce projet est réalisé dans le cadre de mes études à l'ENSIIE.\newline
    Le but est d'implémenter des algorithmes de recherche de 'chemin le plus court', dans des graphes orientés.\newline
    Ce document rapporte mon travail, et explique les choix techniques que j'ai pris.\newline

  \newpage
  \section{Recherche de chemin le plus court}
    Soit (X, A) un graphe orienté. Le recherche du chemin le plus court.....

  \subsection{Parcours en largeur (graphes non-pondérés)}
    On considère ici un graphe où les arcs sont non pondérés.\newline
    'Le chemin le plus court' entre 2 sommets corresponds donc au nombre d'arc minimums necessaire pour les relier par un chemin.\newline
    J'ai representé le graphe par le triplet (n, X, A), avec:
    \begin{itemize}[label=-]
      \item n : Card(X)
      \item X : tableau de n 'sommets'
      \item A : tableau de n * n bits representant les arcs
    \end{itemize}
    
    Le sommet 's' correspond au sommet source (choisie comme origine des distances).
    Un sommet 't' est une structure contenant les informations propres à un sommet:
    \begin{itemize}[label=-]
      \item pathlen : longueur du chemin minimum reliant 't' au sommet 's'
      \item path : tableau de taille 'pathlen' contenant les sommets du chemin
    \end{itemize}
    
    Les arcs sont coddés sur une suite de bits.
    Considérons la bijection:
	F : [0, n - 1]^2 ->   [0, n^2]
	  :    (i, j)    ->  i * n + j
	
    Le p-ième bit vaut 1 ou 0, selon qu'il existe un c

  \subsection{Algorithme de Dijkstra (graphes pondérés)}

    Dummy text

  \newpage
  \section{Application: resolution labyrinthe}

\end{document}
